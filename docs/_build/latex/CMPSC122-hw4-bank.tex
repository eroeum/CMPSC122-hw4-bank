%% Generated by Sphinx.
\def\sphinxdocclass{report}
\documentclass[letterpaper,10pt,english]{sphinxmanual}
\ifdefined\pdfpxdimen
   \let\sphinxpxdimen\pdfpxdimen\else\newdimen\sphinxpxdimen
\fi \sphinxpxdimen=.75bp\relax

\usepackage[utf8]{inputenc}
\ifdefined\DeclareUnicodeCharacter
 \ifdefined\DeclareUnicodeCharacterAsOptional
  \DeclareUnicodeCharacter{"00A0}{\nobreakspace}
  \DeclareUnicodeCharacter{"2500}{\sphinxunichar{2500}}
  \DeclareUnicodeCharacter{"2502}{\sphinxunichar{2502}}
  \DeclareUnicodeCharacter{"2514}{\sphinxunichar{2514}}
  \DeclareUnicodeCharacter{"251C}{\sphinxunichar{251C}}
  \DeclareUnicodeCharacter{"2572}{\textbackslash}
 \else
  \DeclareUnicodeCharacter{00A0}{\nobreakspace}
  \DeclareUnicodeCharacter{2500}{\sphinxunichar{2500}}
  \DeclareUnicodeCharacter{2502}{\sphinxunichar{2502}}
  \DeclareUnicodeCharacter{2514}{\sphinxunichar{2514}}
  \DeclareUnicodeCharacter{251C}{\sphinxunichar{251C}}
  \DeclareUnicodeCharacter{2572}{\textbackslash}
 \fi
\fi
\usepackage{cmap}
\usepackage[T1]{fontenc}
\usepackage{amsmath,amssymb,amstext}
\usepackage{babel}
\usepackage{times}
\usepackage[Bjarne]{fncychap}
\usepackage{sphinx}

\usepackage{geometry}

% Include hyperref last.
\usepackage{hyperref}
% Fix anchor placement for figures with captions.
\usepackage{hypcap}% it must be loaded after hyperref.
% Set up styles of URL: it should be placed after hyperref.
\urlstyle{same}
\addto\captionsenglish{\renewcommand{\contentsname}{Contents:}}

\addto\captionsenglish{\renewcommand{\figurename}{Fig.}}
\addto\captionsenglish{\renewcommand{\tablename}{Table}}
\addto\captionsenglish{\renewcommand{\literalblockname}{Listing}}

\addto\captionsenglish{\renewcommand{\literalblockcontinuedname}{continued from previous page}}
\addto\captionsenglish{\renewcommand{\literalblockcontinuesname}{continues on next page}}

\addto\extrasenglish{\def\pageautorefname{page}}

\setcounter{tocdepth}{1}



\title{CMPSC122-hw4-bank Documentation}
\date{Mar 14, 2018}
\release{}
\author{Eric Roeum}
\newcommand{\sphinxlogo}{\vbox{}}
\renewcommand{\releasename}{}
\makeindex

\begin{document}

\maketitle
\sphinxtableofcontents
\phantomsection\label{\detokenize{index::doc}}

\section{Introduction}
\label{\detokenize{intro:intro}}\label{\detokenize{intro::doc}}
This document discusses how to use this program and how it operates

\subsection{Setup}
\label{\detokenize{intro:setup}}\label{\detokenize{intro::doc}}
This program only requires the python standard library and must be at least version 3.6.4.  To develop any more documentation, sphinx must be installed.  On a linux machine perform the following to setup:
\begin{enumerate}
\item{apt-get update}
\item{apt-get -y upgrade}
\item{apt-get -y install build-essential python3-pip python3-dev}
\item{pip install sphinx (Optional for documentation generation}
\item{git clone https://github.com/eroeum/CMPSC122-hw4-bank.git}
\end{enumerate}

\subsection{Running the Program}
\label{\detokenize{intro:running}}\label{\detokenize{intro::doc}}
To run the program perform the following:
\begin{enumerate}
\item{Go to CMPSC122-hw4-bank directory}
\item{Run the program by running \textbf{bank/core.py -t} (Note: the -t is to run the program in terminal)}
\end{enumerate}
An example of running in a Linux shell is below:
\begin{enumerate}
\item{cd CMPSC122-hw4-bank}
\item{python3 bank/core.py -t}
\end{enumerate}

\newpage

\section{Example}
\label{\detokenize{intro:example}}\label{\detokenize{intro::doc}}
To help go through the program, an example of its functionallity is given below.  Note that anything italicized is entered after each punctuation mark and anything after \# are notes.  Anything in "[]" should be also entered but the value depends after each run.  This can be used as a rudamentary systems acceptance test.
\begin{itemize}
\item{\# Make a Manger \& set-up bank}
\item{:\textit{man}}
\item{:\textit{pass}}
\item{\# Make 3 Customers}
\item{:\textit{new}}
\item{:\textit{new}}
\item{:\textit{cust1}}
\item{:\textit{pass}}
\item{:\textit{new}}
\item{:\textit{new}}
\item{:\textit{cust2}}
\item{:\textit{pass}}
\item{:\textit{new}}
\item{:\textit{new}}
\item{:\textit{cust3}}
\item{:\textit{pass}}
\item{\# Login as the manger}
\item{:\textit{man}}
\item{:\textit{pass}}
\item{\# Make an assistant and bank teller}
\item{>>\textit{make assi1 pass}}
\item{>>\textit{make tell1 pass}}
\item{>>\textit{exit}}
\item{\# Login as a customer}
\item{:\textit{cust1}}
\item{:\textit{pass}}
\item{\# Create some transactions}
\item{>>\textit{balance}\# The balance should be 0}
\item{>>\textit{deposit 100}}
\item{>>\textit{withdrawal 40}}
\item{>>\textit{deposit 10}}
\item{>>\textit{balance}\# The balance should still be 0 until the transaction is approved}
\item{>>\textit{exit}}
\item{\# Login as a bankteller and approve a transaction}
\item{:\textit{assi1}}
\item{:\textit{pass}}
\item{>>\textit{ls} \# A list of all the customers should be shown}
\item{>>\textit{requests }[Cust1 userid] \# get the userid by copying the 8-character code that comes first in the ls feature.  This is the hashed user id for security purposes }
\item{>>\textit{accept} [Cust1 user ID] \textit{1} \# Accepts the first customer's first transaction}
\item{>>\textit{exit}}
\item{\# Check that the transaction went through}
\item{:\textit{cust1}}
\item{:\textit{pass}}
\item{>>\textit{balance} \# The balance should be 100 now}
\item{>>\textit{exit}}
\item{\# Exit out of the program}
\item{:\textit{exit}}
\item{:\textit{exit}}
\end{itemize}
This list goes over the fundamental functions of this program.  Please continue to explore by creating an account and typing "help" for a full list of description and things to do.  For more help, type "help" plus the function name.

\newpage

\section{Caveats \& Fundamentals}
\label{\detokenize{intro:example}}\label{\detokenize{intro::doc}}
The purpose of this program is to simulate a bank.  Of course, because of the level of programming that this was programmed with, the security measures are not industry-worthy and this bank does not, in any shape, accept any real monetary value and only deals with hypothetical values.  As a result, a user can deposit and withdrawal as much money as he/she wants with no penalty.  The users' balance may even drop below 0 to simulate debt, but because of the nature of this bank, there is no penalty for doing such.

The fundamentals of the hierarchy is as follows:
\begin{itemize}
\item{Manager: Head of the bank, can do anything.  There is only one manager since this person/group should share the same account while assistants help the manager.  The only real addition that the manager has opposed to an assistant is creating assistants.}
\item{Assistant: Second in command.  There can be as many assistants that the manager wants.  The assistant can create new bank tellers and do anything a bank teller can do.}
\item{Bank Teller: Lowest administrative user.  The bank teller can only accept requests that a customer has submitted when completing a transaction.}
\item{Customer: No real power in the bank.  Can only submit requests to complete transactions}
\end{itemize}

Additional caveats include that there was no consideration for loans or subaccounts or shared accounts or things of those nature.  Since this is a simply a simulation of a bank with no penalty, constructing a bank with these features requires investigation into penalty and economic theory to do so.  As a result, it may be a future work.  Further when running the program, a "-t" is needed because if a GUI implementation would be created, that would be the main way of running the program.  Lastly there was no consideration for malicious code insertion, since crypotgraphy is beyond our scope.

The password authentication system relies on SHA512 hashing which may not be the best, but is still better than nothing since it allows a one-way authentication system that is blind for both the user and the bank administrators.

\chapter{CMPSC122-hw4-bank}
\label{\detokenize{modules:welcome-to-cmpsc122-hw4-bank-s-documentation}}\label{\detokenize{modules:cmpsc122-hw4-bank}}\label{\detokenize{modules::doc}}

\section{bank package}
\label{\detokenize{bank:bank-package}}\label{\detokenize{bank::doc}}

\subsection{Submodules}
\label{\detokenize{bank:submodules}}

\subsection{bank.assistant module}
\label{\detokenize{bank:module-bank.assistant}}\label{\detokenize{bank:bank-assistant-module}}\index{bank.assistant (module)}\index{Assistant (class in bank.assistant)}

\begin{fulllineitems}
\phantomsection\label{\detokenize{bank:bank.assistant.Assistant}}\pysiglinewithargsret{\sphinxbfcode{\sphinxupquote{class }}\sphinxcode{\sphinxupquote{bank.assistant.}}\sphinxbfcode{\sphinxupquote{Assistant}}}{\emph{userID}, \emph{customers}, \emph{bankTellers}}{}
Bases: \sphinxcode{\sphinxupquote{bankTeller.BankTeller}}

Represents a user with mid-level priveledges
Users having these priveledges are usually bank tellers
\index{createBankTeller() (bank.assistant.Assistant method)}

\begin{fulllineitems}
\phantomsection\label{\detokenize{bank:bank.assistant.Assistant.createBankTeller}}\pysiglinewithargsret{\sphinxbfcode{\sphinxupquote{createBankTeller}}}{\emph{userID}, \emph{customers}}{}
Creates a Bank Teller under this account
\begin{quote}\begin{description}
\item[{Parameters}] \leavevmode\begin{itemize}
\item {} 
\sphinxstyleliteralstrong{\sphinxupquote{userID}} (\sphinxstyleliteralemphasis{\sphinxupquote{string}}) \textendash{} the userID of the assistant

\item {} 
\sphinxstyleliteralstrong{\sphinxupquote{customers}} (\sphinxstyleliteralemphasis{\sphinxupquote{list}}) \textendash{} all customers that assitant can access

\item {} 
\sphinxstyleliteralstrong{\sphinxupquote{bankTellers}} (\sphinxstyleliteralemphasis{\sphinxupquote{list}}) \textendash{} all bank tellers that assistant can access

\end{itemize}

\end{description}\end{quote}

\end{fulllineitems}

\index{viewBankTellers() (bank.assistant.Assistant method)}

\begin{fulllineitems}
\phantomsection\label{\detokenize{bank:bank.assistant.Assistant.viewBankTellers}}\pysiglinewithargsret{\sphinxbfcode{\sphinxupquote{viewBankTellers}}}{}{}
Returns the bank tellers under this account

\end{fulllineitems}


\end{fulllineitems}



\subsection{bank.bankTeller module}
\label{\detokenize{bank:module-bank.bankTeller}}\label{\detokenize{bank:bank-bankteller-module}}\index{bank.bankTeller (module)}\index{BankTeller (class in bank.bankTeller)}

\begin{fulllineitems}
\phantomsection\label{\detokenize{bank:bank.bankTeller.BankTeller}}\pysiglinewithargsret{\sphinxbfcode{\sphinxupquote{class }}\sphinxcode{\sphinxupquote{bank.bankTeller.}}\sphinxbfcode{\sphinxupquote{BankTeller}}}{\emph{userID}, \emph{customers}}{}
Bases: \sphinxcode{\sphinxupquote{customer.Customer}}

Represents a user with mid-level priveledges
Users having these priveledges are usually bank tellers
\index{acceptRequest() (bank.bankTeller.BankTeller method)}

\begin{fulllineitems}
\phantomsection\label{\detokenize{bank:bank.bankTeller.BankTeller.acceptRequest}}\pysiglinewithargsret{\sphinxbfcode{\sphinxupquote{acceptRequest}}}{\emph{customer}, \emph{reqNum}}{}
Accepts a request for a customer
\begin{quote}\begin{description}
\item[{Parameters}] \leavevmode\begin{itemize}
\item {} 
\sphinxstyleliteralstrong{\sphinxupquote{customer}} ({\hyperref[\detokenize{bank:bank.customer.Customer}]{\sphinxcrossref{\sphinxstyleliteralemphasis{\sphinxupquote{Customer}}}}}) \textendash{} Customer with the request

\item {} 
\sphinxstyleliteralstrong{\sphinxupquote{reqNum}} (\sphinxstyleliteralemphasis{\sphinxupquote{int}}) \textendash{} Request number that wishes to be resolved

\end{itemize}

\end{description}\end{quote}

\end{fulllineitems}

\index{addHistory() (bank.bankTeller.BankTeller method)}

\begin{fulllineitems}
\phantomsection\label{\detokenize{bank:bank.bankTeller.BankTeller.addHistory}}\pysiglinewithargsret{\sphinxbfcode{\sphinxupquote{addHistory}}}{}{}
Bank tellers do no actually have any history
All history is written in customers

\end{fulllineitems}

\index{getHistory() (bank.bankTeller.BankTeller method)}

\begin{fulllineitems}
\phantomsection\label{\detokenize{bank:bank.bankTeller.BankTeller.getHistory}}\pysiglinewithargsret{\sphinxbfcode{\sphinxupquote{getHistory}}}{}{}
Bank tellers do no actually have history
All history is written in customers

\end{fulllineitems}

\index{getRequests() (bank.bankTeller.BankTeller method)}

\begin{fulllineitems}
\phantomsection\label{\detokenize{bank:bank.bankTeller.BankTeller.getRequests}}\pysiglinewithargsret{\sphinxbfcode{\sphinxupquote{getRequests}}}{}{}
Bank tellers do no actually have requests

\end{fulllineitems}

\index{getSubaccounts() (bank.bankTeller.BankTeller method)}

\begin{fulllineitems}
\phantomsection\label{\detokenize{bank:bank.bankTeller.BankTeller.getSubaccounts}}\pysiglinewithargsret{\sphinxbfcode{\sphinxupquote{getSubaccounts}}}{}{}
Bank tellers do no actually have a sub accounts

\end{fulllineitems}

\index{getValue() (bank.bankTeller.BankTeller method)}

\begin{fulllineitems}
\phantomsection\label{\detokenize{bank:bank.bankTeller.BankTeller.getValue}}\pysiglinewithargsret{\sphinxbfcode{\sphinxupquote{getValue}}}{}{}
Bank tellers do no actually have a value

\end{fulllineitems}

\index{removeRequest() (bank.bankTeller.BankTeller method)}

\begin{fulllineitems}
\phantomsection\label{\detokenize{bank:bank.bankTeller.BankTeller.removeRequest}}\pysiglinewithargsret{\sphinxbfcode{\sphinxupquote{removeRequest}}}{}{}
Bank tellers do no actually have any requests

\end{fulllineitems}

\index{requestDelta() (bank.bankTeller.BankTeller method)}

\begin{fulllineitems}
\phantomsection\label{\detokenize{bank:bank.bankTeller.BankTeller.requestDelta}}\pysiglinewithargsret{\sphinxbfcode{\sphinxupquote{requestDelta}}}{}{}
Bank tellers do no actually have a value

\end{fulllineitems}

\index{viewCustomers() (bank.bankTeller.BankTeller method)}

\begin{fulllineitems}
\phantomsection\label{\detokenize{bank:bank.bankTeller.BankTeller.viewCustomers}}\pysiglinewithargsret{\sphinxbfcode{\sphinxupquote{viewCustomers}}}{}{}
Returns the customers under this account

\end{fulllineitems}

\index{viewRequests() (bank.bankTeller.BankTeller method)}

\begin{fulllineitems}
\phantomsection\label{\detokenize{bank:bank.bankTeller.BankTeller.viewRequests}}\pysiglinewithargsret{\sphinxbfcode{\sphinxupquote{viewRequests}}}{\emph{customer}}{}
Retrieves the requests held within the customer
\begin{quote}\begin{description}
\item[{Parameters}] \leavevmode
\sphinxstyleliteralstrong{\sphinxupquote{customer}} ({\hyperref[\detokenize{bank:bank.customer.Customer}]{\sphinxcrossref{\sphinxstyleliteralemphasis{\sphinxupquote{Customer}}}}}) \textendash{} Customer with requests

\end{description}\end{quote}

\end{fulllineitems}


\end{fulllineitems}



\subsection{bank.core module}
\label{\detokenize{bank:bank-core-module}}\label{\detokenize{bank:module-bank.core}}\index{bank.core (module)}\index{main() (in module bank.core)}

\begin{fulllineitems}
\phantomsection\label{\detokenize{bank:bank.core.main}}\pysiglinewithargsret{\sphinxcode{\sphinxupquote{bank.core.}}\sphinxbfcode{\sphinxupquote{main}}}{\emph{arg}}{}
Runs the core of the program
Integrates all other python scripts/modules
Use the core.py to run the Bank python script
\begin{quote}\begin{description}
\item[{Parameters}] \leavevmode
\sphinxstyleliteralstrong{\sphinxupquote{arg}} (\sphinxstyleliteralemphasis{\sphinxupquote{list}}) \textendash{} list of arguments in terminal

\end{description}\end{quote}

\end{fulllineitems}



\subsection{bank.customer module}
\label{\detokenize{bank:bank-customer-module}}\label{\detokenize{bank:module-bank.customer}}\index{bank.customer (module)}\index{Customer (class in bank.customer)}

\begin{fulllineitems}
\phantomsection\label{\detokenize{bank:bank.customer.Customer}}\pysiglinewithargsret{\sphinxbfcode{\sphinxupquote{class }}\sphinxcode{\sphinxupquote{bank.customer.}}\sphinxbfcode{\sphinxupquote{Customer}}}{\emph{value}, \emph{userID\_owner}, \emph{userID\_users={[}{]}}}{}
Bases: \sphinxcode{\sphinxupquote{object}}

Represents a user with low-level priveledges
Users only having these priveledges are usually customers
\index{addHistory() (bank.customer.Customer method)}

\begin{fulllineitems}
\phantomsection\label{\detokenize{bank:bank.customer.Customer.addHistory}}\pysiglinewithargsret{\sphinxbfcode{\sphinxupquote{addHistory}}}{\emph{event}}{}
Adds an event to the history
\begin{quote}\begin{description}
\item[{Parameters}] \leavevmode
\sphinxstyleliteralstrong{\sphinxupquote{event}} (\sphinxstyleliteralemphasis{\sphinxupquote{string}}) \textendash{} Event that has take place (only accepted requests)

\end{description}\end{quote}

\end{fulllineitems}

\index{getHistory() (bank.customer.Customer method)}

\begin{fulllineitems}
\phantomsection\label{\detokenize{bank:bank.customer.Customer.getHistory}}\pysiglinewithargsret{\sphinxbfcode{\sphinxupquote{getHistory}}}{}{}
Retrieves history of this account

\end{fulllineitems}

\index{getOwner() (bank.customer.Customer method)}

\begin{fulllineitems}
\phantomsection\label{\detokenize{bank:bank.customer.Customer.getOwner}}\pysiglinewithargsret{\sphinxbfcode{\sphinxupquote{getOwner}}}{}{}
Retrieves owner of the bank

\end{fulllineitems}

\index{getRequests() (bank.customer.Customer method)}

\begin{fulllineitems}
\phantomsection\label{\detokenize{bank:bank.customer.Customer.getRequests}}\pysiglinewithargsret{\sphinxbfcode{\sphinxupquote{getRequests}}}{}{}
Retrieves request of account

\end{fulllineitems}

\index{getSubaccounts() (bank.customer.Customer method)}

\begin{fulllineitems}
\phantomsection\label{\detokenize{bank:bank.customer.Customer.getSubaccounts}}\pysiglinewithargsret{\sphinxbfcode{\sphinxupquote{getSubaccounts}}}{}{}
Retrives accounts owned by this account

\end{fulllineitems}

\index{getUsers() (bank.customer.Customer method)}

\begin{fulllineitems}
\phantomsection\label{\detokenize{bank:bank.customer.Customer.getUsers}}\pysiglinewithargsret{\sphinxbfcode{\sphinxupquote{getUsers}}}{}{}
Retrieves priveledged users of the bank

\end{fulllineitems}

\index{getValue() (bank.customer.Customer method)}

\begin{fulllineitems}
\phantomsection\label{\detokenize{bank:bank.customer.Customer.getValue}}\pysiglinewithargsret{\sphinxbfcode{\sphinxupquote{getValue}}}{}{}
Retrieves value held within bank

\end{fulllineitems}

\index{removeRequest() (bank.customer.Customer method)}

\begin{fulllineitems}
\phantomsection\label{\detokenize{bank:bank.customer.Customer.removeRequest}}\pysiglinewithargsret{\sphinxbfcode{\sphinxupquote{removeRequest}}}{\emph{reqNum}}{}
Removes request from request list
\begin{quote}\begin{description}
\item[{Parameters}] \leavevmode
\sphinxstyleliteralstrong{\sphinxupquote{reqNum}} (\sphinxstyleliteralemphasis{\sphinxupquote{int}}) \textendash{} Index of request

\end{description}\end{quote}

\end{fulllineitems}

\index{requestDelta() (bank.customer.Customer method)}

\begin{fulllineitems}
\phantomsection\label{\detokenize{bank:bank.customer.Customer.requestDelta}}\pysiglinewithargsret{\sphinxbfcode{\sphinxupquote{requestDelta}}}{\emph{delta}}{}
Rquests the withdrawl or deposit into customer’s account
\begin{quote}\begin{description}
\item[{Parameters}] \leavevmode
\sphinxstyleliteralstrong{\sphinxupquote{delta}} (\sphinxstyleliteralemphasis{\sphinxupquote{float}}) \textendash{} Amount to request funds

\end{description}\end{quote}

\end{fulllineitems}


\end{fulllineitems}



\subsection{bank.manager module}
\label{\detokenize{bank:module-bank.manager}}\label{\detokenize{bank:bank-manager-module}}\index{bank.manager (module)}\index{Manager (class in bank.manager)}

\begin{fulllineitems}
\phantomsection\label{\detokenize{bank:bank.manager.Manager}}\pysiglinewithargsret{\sphinxbfcode{\sphinxupquote{class }}\sphinxcode{\sphinxupquote{bank.manager.}}\sphinxbfcode{\sphinxupquote{Manager}}}{\emph{userID}, \emph{customers}, \emph{bankTellers}, \emph{assistants}}{}
Bases: \sphinxcode{\sphinxupquote{assistant.Assistant}}

Represents a user with high-level priveledges
Users having these priveledges are usually managers
\index{createAssistant() (bank.manager.Manager method)}

\begin{fulllineitems}
\phantomsection\label{\detokenize{bank:bank.manager.Manager.createAssistant}}\pysiglinewithargsret{\sphinxbfcode{\sphinxupquote{createAssistant}}}{\emph{userID}, \emph{customers}, \emph{bankTellers}}{}
Creates an assistant object
\begin{quote}\begin{description}
\item[{Parameters}] \leavevmode\begin{itemize}
\item {} 
\sphinxstyleliteralstrong{\sphinxupquote{userID}} (\sphinxstyleliteralemphasis{\sphinxupquote{string}}) \textendash{} the userID of the assistant

\item {} 
\sphinxstyleliteralstrong{\sphinxupquote{customers}} (\sphinxstyleliteralemphasis{\sphinxupquote{list}}) \textendash{} all customers that assistant can access

\item {} 
\sphinxstyleliteralstrong{\sphinxupquote{bankTellers}} (\sphinxstyleliteralemphasis{\sphinxupquote{list}}) \textendash{} all bank tellers that assistant can access

\end{itemize}

\end{description}\end{quote}

\end{fulllineitems}

\index{viewAssistants() (bank.manager.Manager method)}

\begin{fulllineitems}
\phantomsection\label{\detokenize{bank:bank.manager.Manager.viewAssistants}}\pysiglinewithargsret{\sphinxbfcode{\sphinxupquote{viewAssistants}}}{}{}
Returns the assistants under this class

\end{fulllineitems}


\end{fulllineitems}



\subsection{bank.passwordAuthentication module}
\label{\detokenize{bank:module-bank.passwordAuthentication}}\label{\detokenize{bank:bank-passwordauthentication-module}}\index{bank.passwordAuthentication (module)}\index{Password (class in bank.passwordAuthentication)}

\begin{fulllineitems}
\phantomsection\label{\detokenize{bank:bank.passwordAuthentication.Password}}\pysiglinewithargsret{\sphinxbfcode{\sphinxupquote{class }}\sphinxcode{\sphinxupquote{bank.passwordAuthentication.}}\sphinxbfcode{\sphinxupquote{Password}}}{\emph{approvedUsers}}{}
Bases: \sphinxcode{\sphinxupquote{object}}
\index{addAutheticatedUser() (bank.passwordAuthentication.Password method)}

\begin{fulllineitems}
\phantomsection\label{\detokenize{bank:bank.passwordAuthentication.Password.addAutheticatedUser}}\pysiglinewithargsret{\sphinxbfcode{\sphinxupquote{addAutheticatedUser}}}{\emph{userid}, \emph{password}}{}
Adds autheticated user using UUID and SHA512 password hashing
\begin{quote}\begin{description}
\item[{Parameters}] \leavevmode\begin{itemize}
\item {} 
\sphinxstyleliteralstrong{\sphinxupquote{userid}} (\sphinxstyleliteralemphasis{\sphinxupquote{string}}) \textendash{} User ID that user has chosen

\item {} 
\sphinxstyleliteralstrong{\sphinxupquote{password}} (\sphinxstyleliteralemphasis{\sphinxupquote{string}}) \textendash{} Password to attempt to authenticate userid

\end{itemize}

\end{description}\end{quote}

\end{fulllineitems}

\index{authenticate\_password() (bank.passwordAuthentication.Password method)}

\begin{fulllineitems}
\phantomsection\label{\detokenize{bank:bank.passwordAuthentication.Password.authenticate_password}}\pysiglinewithargsret{\sphinxbfcode{\sphinxupquote{authenticate\_password}}}{\emph{userid}, \emph{password}}{}
Autheticates Passwords using UUID and SHA512 password hashing
\begin{quote}\begin{description}
\item[{Parameters}] \leavevmode\begin{itemize}
\item {} 
\sphinxstyleliteralstrong{\sphinxupquote{userid}} (\sphinxstyleliteralemphasis{\sphinxupquote{string}}) \textendash{} User ID that user has chosen

\item {} 
\sphinxstyleliteralstrong{\sphinxupquote{password}} (\sphinxstyleliteralemphasis{\sphinxupquote{string}}) \textendash{} Password to attempt to authenticate userid

\end{itemize}

\end{description}\end{quote}

\end{fulllineitems}

\index{hashUsername() (bank.passwordAuthentication.Password method)}

\begin{fulllineitems}
\phantomsection\label{\detokenize{bank:bank.passwordAuthentication.Password.hashUsername}}\pysiglinewithargsret{\sphinxbfcode{\sphinxupquote{hashUsername}}}{\emph{userID}}{}
Hashes provided username for hased user id
\begin{quote}\begin{description}
\item[{Parameters}] \leavevmode
\sphinxstyleliteralstrong{\sphinxupquote{userid}} \textendash{} User ID that user has chosen

\end{description}\end{quote}

\end{fulllineitems}

\index{readEncrypted() (bank.passwordAuthentication.Password method)}

\begin{fulllineitems}
\phantomsection\label{\detokenize{bank:bank.passwordAuthentication.Password.readEncrypted}}\pysiglinewithargsret{\sphinxbfcode{\sphinxupquote{readEncrypted}}}{\emph{filename}, \emph{dest='./'}}{}
Imports all usernames and passwords from a txt file
File must be formatted using \_writeEncrypted function
NOTE: All previous passwords will be deleted and replaced
\begin{quote}\begin{description}
\item[{Parameters}] \leavevmode\begin{itemize}
\item {} 
\sphinxstyleliteralstrong{\sphinxupquote{filename}} (\sphinxstyleliteralemphasis{\sphinxupquote{string}}) \textendash{} filename of txt file

\item {} 
\sphinxstyleliteralstrong{\sphinxupquote{dest}} (\sphinxstyleliteralemphasis{\sphinxupquote{string}}) \textendash{} location to write txt file

\end{itemize}

\end{description}\end{quote}

\end{fulllineitems}

\index{writeEncrypted() (bank.passwordAuthentication.Password method)}

\begin{fulllineitems}
\phantomsection\label{\detokenize{bank:bank.passwordAuthentication.Password.writeEncrypted}}\pysiglinewithargsret{\sphinxbfcode{\sphinxupquote{writeEncrypted}}}{\emph{filename}, \emph{dest='./'}}{}
Writes all usernames and passwords to a txt file
\begin{quote}\begin{description}
\item[{Parameters}] \leavevmode\begin{itemize}
\item {} 
\sphinxstyleliteralstrong{\sphinxupquote{filename}} (\sphinxstyleliteralemphasis{\sphinxupquote{string}}) \textendash{} filename of txt file

\item {} 
\sphinxstyleliteralstrong{\sphinxupquote{dest}} (\sphinxstyleliteralemphasis{\sphinxupquote{string}}) \textendash{} location to write txt file

\end{itemize}

\end{description}\end{quote}

\end{fulllineitems}


\end{fulllineitems}



\subsection{bank.terminalFunctions module}
\label{\detokenize{bank:bank-terminalfunctions-module}}\label{\detokenize{bank:module-bank.terminalFunctions}}\index{bank.terminalFunctions (module)}\index{acceptRequest() (in module bank.terminalFunctions)}

\begin{fulllineitems}
\phantomsection\label{\detokenize{bank:bank.terminalFunctions.acceptRequest}}\pysiglinewithargsret{\sphinxcode{\sphinxupquote{bank.terminalFunctions.}}\sphinxbfcode{\sphinxupquote{acceptRequest}}}{\emph{user}, \emph{customer}, \emph{reqNum}}{}
Accepts a customer’s request
\begin{quote}\begin{description}
\item[{Parameters}] \leavevmode\begin{itemize}
\item {} 
\sphinxstyleliteralstrong{\sphinxupquote{user}} (\sphinxstyleliteralemphasis{\sphinxupquote{class}}) \textendash{} The current user

\item {} 
\sphinxstyleliteralstrong{\sphinxupquote{customer}} (\sphinxstyleliteralemphasis{\sphinxupquote{customer}}) \textendash{} Customer that is being accepted

\item {} 
\sphinxstyleliteralstrong{\sphinxupquote{reqNum}} (\sphinxstyleliteralemphasis{\sphinxupquote{int}}) \textendash{} Request number correponding to the request

\end{itemize}

\end{description}\end{quote}

\end{fulllineitems}

\index{balance() (in module bank.terminalFunctions)}

\begin{fulllineitems}
\phantomsection\label{\detokenize{bank:bank.terminalFunctions.balance}}\pysiglinewithargsret{\sphinxcode{\sphinxupquote{bank.terminalFunctions.}}\sphinxbfcode{\sphinxupquote{balance}}}{\emph{person}}{}
Shows balance in the account

\end{fulllineitems}

\index{clear() (in module bank.terminalFunctions)}

\begin{fulllineitems}
\phantomsection\label{\detokenize{bank:bank.terminalFunctions.clear}}\pysiglinewithargsret{\sphinxcode{\sphinxupquote{bank.terminalFunctions.}}\sphinxbfcode{\sphinxupquote{clear}}}{\emph{platf}}{}
\end{fulllineitems}

\index{deposit() (in module bank.terminalFunctions)}

\begin{fulllineitems}
\phantomsection\label{\detokenize{bank:bank.terminalFunctions.deposit}}\pysiglinewithargsret{\sphinxcode{\sphinxupquote{bank.terminalFunctions.}}\sphinxbfcode{\sphinxupquote{deposit}}}{\emph{person}, \emph{value\_to\_add}}{}
Deposits value into the account
\begin{quote}\begin{description}
\item[{Parameters}] \leavevmode
\sphinxstyleliteralstrong{\sphinxupquote{value\_to\_add}} (\sphinxstyleliteralemphasis{\sphinxupquote{float}}) \textendash{} Value to be added in funds

\end{description}\end{quote}

\end{fulllineitems}

\index{exit() (in module bank.terminalFunctions)}

\begin{fulllineitems}
\phantomsection\label{\detokenize{bank:bank.terminalFunctions.exit}}\pysiglinewithargsret{\sphinxcode{\sphinxupquote{bank.terminalFunctions.}}\sphinxbfcode{\sphinxupquote{exit}}}{}{}
Exits the program
This is a “fake” function that only confirms exit
Clears screen after exit

\end{fulllineitems}

\index{help() (in module bank.terminalFunctions)}

\begin{fulllineitems}
\phantomsection\label{\detokenize{bank:bank.terminalFunctions.help}}\pysiglinewithargsret{\sphinxcode{\sphinxupquote{bank.terminalFunctions.}}\sphinxbfcode{\sphinxupquote{help}}}{\emph{accountType}, \emph{helpFunc=''}}{}
Help menu for giving users all commands with short description
\begin{quote}\begin{description}
\item[{Parameters}] \leavevmode\begin{itemize}
\item {} 
\sphinxstyleliteralstrong{\sphinxupquote{accountType}} (\sphinxstyleliteralemphasis{\sphinxupquote{string}}) \textendash{} Type of account

\item {} 
\sphinxstyleliteralstrong{\sphinxupquote{helpFunc}} (\sphinxstyleliteralemphasis{\sphinxupquote{string}}) \textendash{} Option functional to get more info

\end{itemize}

\end{description}\end{quote}

\end{fulllineitems}

\index{history() (in module bank.terminalFunctions)}

\begin{fulllineitems}
\phantomsection\label{\detokenize{bank:bank.terminalFunctions.history}}\pysiglinewithargsret{\sphinxcode{\sphinxupquote{bank.terminalFunctions.}}\sphinxbfcode{\sphinxupquote{history}}}{\emph{customer}}{}
Prints the history of a customer
\begin{quote}\begin{description}
\item[{Parameters}] \leavevmode
\sphinxstyleliteralstrong{\sphinxupquote{customer}} (\sphinxstyleliteralemphasis{\sphinxupquote{Class}}) \textendash{} customer with desired history

\end{description}\end{quote}

\end{fulllineitems}

\index{ls() (in module bank.terminalFunctions)}

\begin{fulllineitems}
\phantomsection\label{\detokenize{bank:bank.terminalFunctions.ls}}\pysiglinewithargsret{\sphinxcode{\sphinxupquote{bank.terminalFunctions.}}\sphinxbfcode{\sphinxupquote{ls}}}{\emph{accountType}, \emph{account}, \emph{customers}}{}
Lists all accounts accessible
\begin{quote}\begin{description}
\item[{Parameters}] \leavevmode\begin{itemize}
\item {} 
\sphinxstyleliteralstrong{\sphinxupquote{accountType}} (\sphinxstyleliteralemphasis{\sphinxupquote{string}}) \textendash{} type of account

\item {} 
\sphinxstyleliteralstrong{\sphinxupquote{account}} ({\hyperref[\detokenize{bank:bank.customer.Customer}]{\sphinxcrossref{\sphinxstyleliteralemphasis{\sphinxupquote{Customer}}}}}) \textendash{} The account that is being inspected

\end{itemize}

\end{description}\end{quote}

\end{fulllineitems}

\index{make() (in module bank.terminalFunctions)}

\begin{fulllineitems}
\phantomsection\label{\detokenize{bank:bank.terminalFunctions.make}}\pysiglinewithargsret{\sphinxcode{\sphinxupquote{bank.terminalFunctions.}}\sphinxbfcode{\sphinxupquote{make}}}{\emph{genesis}, \emph{accountType}, \emph{userID}, \emph{users}}{}
Create an account
\begin{quote}\begin{description}
\item[{Parameters}] \leavevmode\begin{itemize}
\item {} 
\sphinxstyleliteralstrong{\sphinxupquote{genesis}} ({\hyperref[\detokenize{bank:bank.customer.Customer}]{\sphinxcrossref{\sphinxstyleliteralemphasis{\sphinxupquote{Customer}}}}}) \textendash{} Original account

\item {} 
\sphinxstyleliteralstrong{\sphinxupquote{accountType}} (\sphinxstyleliteralemphasis{\sphinxupquote{string}}) \textendash{} type of account

\item {} 
\sphinxstyleliteralstrong{\sphinxupquote{userID}} (\sphinxstyleliteralemphasis{\sphinxupquote{string}}) \textendash{} user id of the account

\end{itemize}

\end{description}\end{quote}

\end{fulllineitems}

\index{requests() (in module bank.terminalFunctions)}

\begin{fulllineitems}
\phantomsection\label{\detokenize{bank:bank.terminalFunctions.requests}}\pysiglinewithargsret{\sphinxcode{\sphinxupquote{bank.terminalFunctions.}}\sphinxbfcode{\sphinxupquote{requests}}}{\emph{user}, \emph{customer}}{}
Lists all requests of the customer
\begin{quote}\begin{description}
\item[{Parameters}] \leavevmode\begin{itemize}
\item {} 
\sphinxstyleliteralstrong{\sphinxupquote{user}} (\sphinxstyleliteralemphasis{\sphinxupquote{class}}) \textendash{} The current user

\item {} 
\sphinxstyleliteralstrong{\sphinxupquote{customer}} (\sphinxstyleliteralemphasis{\sphinxupquote{customer}}) \textendash{} Customer that will be viewed

\end{itemize}

\end{description}\end{quote}

\end{fulllineitems}

\index{whoami() (in module bank.terminalFunctions)}

\begin{fulllineitems}
\phantomsection\label{\detokenize{bank:bank.terminalFunctions.whoami}}\pysiglinewithargsret{\sphinxcode{\sphinxupquote{bank.terminalFunctions.}}\sphinxbfcode{\sphinxupquote{whoami}}}{\emph{person}}{}
Return the owner’s ID

\end{fulllineitems}

\index{withdrawal() (in module bank.terminalFunctions)}

\begin{fulllineitems}
\phantomsection\label{\detokenize{bank:bank.terminalFunctions.withdrawal}}\pysiglinewithargsret{\sphinxcode{\sphinxupquote{bank.terminalFunctions.}}\sphinxbfcode{\sphinxupquote{withdrawal}}}{\emph{person}, \emph{value\_to\_deduct}}{}
Withdrawals value from the account
\begin{quote}\begin{description}
\item[{Parameters}] \leavevmode
\sphinxstyleliteralstrong{\sphinxupquote{value\_to\_deduct}} (\sphinxstyleliteralemphasis{\sphinxupquote{float}}) \textendash{} Value to be withdrawled in funds

\end{description}\end{quote}

\end{fulllineitems}



\subsection{bank.terminalInterface module}
\label{\detokenize{bank:module-bank.terminalInterface}}\label{\detokenize{bank:bank-terminalinterface-module}}\index{bank.terminalInterface (module)}\index{displayInterface() (in module bank.terminalInterface)}

\begin{fulllineitems}
\phantomsection\label{\detokenize{bank:bank.terminalInterface.displayInterface}}\pysiglinewithargsret{\sphinxcode{\sphinxupquote{bank.terminalInterface.}}\sphinxbfcode{\sphinxupquote{displayInterface}}}{}{}
Terminal-Based banking application
\begin{quote}\begin{description}
\item[{Parameters}] \leavevmode
\sphinxstyleliteralstrong{\sphinxupquote{customers}} (\sphinxstyleliteralemphasis{\sphinxupquote{dict}}) \textendash{} dictionary of all created customers

\item[{Params users}] \leavevmode
Class with all developed user-password authentication

\end{description}\end{quote}

\end{fulllineitems}



\subsection{Module contents}
\label{\detokenize{bank:module-contents}}\label{\detokenize{bank:module-bank}}\index{bank (module)}

\section{tests package}
\label{\detokenize{tests:tests-package}}\label{\detokenize{tests::doc}}

\subsection{Module contents}
\label{\detokenize{tests:module-tests}}\label{\detokenize{tests:module-contents}}\index{tests (module)}

\chapter{Indices and tables}
\label{\detokenize{index:indices-and-tables}}\begin{itemize}
\item {} 
\DUrole{xref,std,std-ref}{genindex}

\item {} 
\DUrole{xref,std,std-ref}{modindex}

\item {} 
\DUrole{xref,std,std-ref}{search}

\end{itemize}


\renewcommand{\indexname}{Python Module Index}
\begin{sphinxtheindex}
\def\bigletter#1{{\Large\sffamily#1}\nopagebreak\vspace{1mm}}
\bigletter{b}
\item {\sphinxstyleindexentry{bank}}\sphinxstyleindexpageref{bank:\detokenize{module-bank}}
\item {\sphinxstyleindexentry{bank.assistant}}\sphinxstyleindexpageref{bank:\detokenize{module-bank.assistant}}
\item {\sphinxstyleindexentry{bank.bankTeller}}\sphinxstyleindexpageref{bank:\detokenize{module-bank.bankTeller}}
\item {\sphinxstyleindexentry{bank.core}}\sphinxstyleindexpageref{bank:\detokenize{module-bank.core}}
\item {\sphinxstyleindexentry{bank.customer}}\sphinxstyleindexpageref{bank:\detokenize{module-bank.customer}}
\item {\sphinxstyleindexentry{bank.manager}}\sphinxstyleindexpageref{bank:\detokenize{module-bank.manager}}
\item {\sphinxstyleindexentry{bank.passwordAuthentication}}\sphinxstyleindexpageref{bank:\detokenize{module-bank.passwordAuthentication}}
\item {\sphinxstyleindexentry{bank.terminalFunctions}}\sphinxstyleindexpageref{bank:\detokenize{module-bank.terminalFunctions}}
\item {\sphinxstyleindexentry{bank.terminalInterface}}\sphinxstyleindexpageref{bank:\detokenize{module-bank.terminalInterface}}
\indexspace
\bigletter{t}
\item {\sphinxstyleindexentry{tests}}\sphinxstyleindexpageref{tests:\detokenize{module-tests}}
\end{sphinxtheindex}

\renewcommand{\indexname}{Index}
\printindex
\end{document}